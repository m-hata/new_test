\setcounter{secnumdepth}{3}

\makeatletter

%%% 見出し rmsection: (1)..., (2)..., (3)...
%\newcounter{rmsection}[subsubsection]
%\renewcommand{\thermsection}{(\arabic{rmsection})}
%\let\rmsectionmark\@gobble
%\newcommand*{\l@rmsection}{\@dottedtocline{3}{10em}{5em}}
%\newcommand{\rmsection}{\@startsection{rmsection}{3}{\z@}%
%  {3.25ex \@plus 1ex \@minus .2ex}% 見出しの前の縦空白
%  {.5\Cvs}% 見出しの後の縦空白
%  {\reset@font\normalsize\bfseries}}% 見出しのスタイル
%\def\toclevel@rmsection{4} % for hyperref

%% 見出し rmproject: (i)..., (ii)..., (iii)...
%\newcounter{rmproject}[rmsection]
\newcounter{rmproject}[subsection]
\renewcommand{\thermproject}{(\roman{rmproject})}
\let\rmprojectmark\@gobble
\newcommand*{\l@rmproject}{\@dottedtocline{3}{12em}{6em}}
\newcommand{\rmproject}{\@startsection{rmproject}{3}{\z@}%
  {3.25ex \@plus 1ex \@minus .2ex}% 見出しの前の縦空白
  {1sp minus 2sp}% 見出しの後、縦空白なし
  {\reset@font\normalsize\bfseries}}% 見出しのスタイル
  %{\reset@font\normalsize\normalfont}}% 見出しのスタイル
\def\toclevel@rmproject{4} % for hyperref

% 見出し sragraphl: 改行ありparagraph
\newcommand\paragraphl[1]{\paragraph{#1}\mbox{}\par}

% 見出し subparagraphl: 改行ありsubparagraph
\newcommand\subparagraphl[1]{\subparagraph{#1}\mbox{}\par}
%\newcommand{\subparagraphl}{\@startsection{subparagraph}{5}{\z@}%
%   {3.25ex \@plus 1ex \@minus .2ex}%
%   {1sp minus 2sp}% 見出しの後、改行、縦空白なし
%   {\reset@font\normalsize\bfseries}}
%%  {\reset@font\normalsize\normalfont}}  %見出しはノーマルフォント

% jsbook.clsの修正
%   - subusubsectionの見出し後に縦空白を増やす
%   - paragraphの見出し前に■を入れない
\renewcommand{\subsubsection}{\@startsection{subsubsection}{3}{\z@}%
   {1.5\Cvs \@plus.5\Cvs \@minus.2\Cvs}%
   {.5\Cvs \@plus.3\Cvs}%
   {\normalfont\normalsize\headfont}}
\renewcommand{\paragraph}{\@startsection{paragraph}{4}{\z@}%
   {0.5\Cvs \@plus.5\Cdp \@minus.2\Cdp}%
   {-1zw}% 改行せず 1zw のアキ
   {\normalfont\normalsize\headfont}}


%% 図番号と表番号の形式をWord文書と同じにする
%\def\thefigure{%
% \ifnum\value{subsection}>0
%   {\thechapter.\arabic{section}.\arabic{subsection}-\arabic{figure}}
% \else
%   {\thechapter.\arabic{section}-\arabic{figure}}
% \fi}
%\@addtoreset{figure}{subsection}
%\def\thetable{%
% \ifnum\value{subsection}>0
%   {\thechapter.\arabic{section}.\arabic{subsection}-\arabic{table}}
% \else
%   {\thechapter.\arabic{section}-\arabic{table}}
% \fi}
%\@addtoreset{table}{subsection}


%% 「参考文献」をsubsection*環境で出力
\renewenvironment{thebibliography}[1]{%
  \global\let\presectionname\relax
  \global\let\postsectionname\relax
%%\chapter*{\bibname}\@mkboth{\bibname}{}%
%%\addcontentsline{toc}{chapter}{\bibname}%
  \subsection*{\bibname}%
  \addcontentsline{toc}{subsection}{\bibname}%
   \list{\@biblabel{\@arabic\c@enumiv}}%
        {\settowidth\labelwidth{\@biblabel{#1}}%
         \leftmargin\labelwidth
         \advance\leftmargin\labelsep
         \@openbib@code
         \usecounter{enumiv}%
         \let\p@enumiv\@empty
         \renewcommand\theenumiv{\@arabic\c@enumiv}}%
   \sloppy
   \clubpenalty4000
   \@clubpenalty\clubpenalty
   \widowpenalty4000%
   \sfcode`\.\@m}
  {\def\@noitemerr
    {\@latex@warning{Empty `thebibliography' environment}}%
   \endlist}
\makeatother

%% 出典
\newcommand{\出典}[1]{\small 出典:{#1}}

%% 用語集
\newcommand{\glossarySection}[1]{\rowcolor[gray]{.90}\multicolumn{2}{|l|}{#1}\\\hline}
\newcommand{\glossaryItem}[3]{#1 & #3 \\\hline}  % #2は「ふりがな」
\newenvironment{用語集}
 {\renewcommand{\item}{\glossaryItem}
  \renewcommand{\section}{\glossarySection}
  \begin{longtable}[c]{|m{10em}|m{30em}|}
  \hline
  \rowcolor[gray]{.75} \textbf{用語} & \textbf{解説} \\ \hline \endhead
  \multicolumn{2}{r@{}}{\footnotesize (次ページに続く)} \endfoot
  \endlastfoot}
 {\end{longtable}}


%% 執筆者一覧
\newcommand{\authorsSection}[1]{\rowcolor[gray]{.75}\multicolumn{3}{|l|}{\textbf{#1}}\\\hline}
\newcommand{\authorsSubection}[1]{\rowcolor[gray]{.90}\multicolumn{3}{|l|}{#1}\\\hline}
\newcommand{\authorsItem}[3]{#1 & #2 & #3\\\hline}
\newenvironment{執筆者一覧}
 {\renewcommand{\item}{\authorsItem}
  \renewcommand{\section}{\authorsSection}
  \renewcommand{\subsection}{\authorsSubection}
  \begin{longtable}[c]{|m{5em}|m{27em}|m{8em}|}
  \hline
  \textbf{氏名} & \textbf{所属} & \textbf{役職} \\ \hline \endhead
  \multicolumn{3}{r@{}}{\footnotesize (次ページに続く)} \endfoot
  \endlastfoot}
 {\end{longtable}}


% 丸数字
\newcommand{\MARU}[1]{{\ooalign{\hfil#1\/\hfil\crcr
                  \raise.167ex\hbox{\mathhexbox20D}}}}

% 要求性能表
\newcommand{\ReqSpecRotate}[1]{\rotatebox[origin=c]{85}{\hspace{.5em}\parbox{8em}{#1}}}

\newcommand\ReqSpecHead{%
&
\ReqSpecRotate{要求性能\newline (PFLOPS)} &
\ReqSpecRotate{要求メモリバンド幅\newline (PB/s)} &
\ReqSpecRotate{メモリ量/ケース\newline (PB)} &
\ReqSpecRotate{ストレージ量/ケース\newline (PB)} &
\ReqSpecRotate{計算時間/ケース\newline (hour)} & 
\ReqSpecRotate{ケース数} & 
\ReqSpecRotate{総演算量\newline (EFLOP)} &
\ReqSpecRotate{概要と計算手法} &
\ReqSpecRotate{問題規模} 
\\ \midrule
}

\newenvironment{要求性能表}[3]
{%
  \begin{ThreePartTable}
    \renewcommand\TPTminimum{\textwidth}
    \centering 
    \scriptsize
    #2
   %\begin{longtable}{m{6em}|l|l|l|l|l|l|l|m{6em}|m{8em}}
    \begin{longtable}{m{6em}|lllllllm{6em}m{8em}}
      \caption*{\normalsize 「#1」要求性能表} \\
      \toprule
      \ReqSpecHead
    \endfirsthead
      \caption*{\normalsize 「#1」要求性能表(続き)} \\
      \toprule
      \ReqSpecHead
    \endhead
      \multicolumn{10}{r@{}}{\scriptsize (次ページに続く)}
    \endfoot
      \bottomrule
      #3
    \endlastfoot 
}
{%
    \end{longtable}
  \end{ThreePartTable}
}
